\documentclass[openany, twoside, a4paper, 12pt]{extbook}

\usepackage[utf8]{inputenc}
\usepackage{rotating}
\usepackage[russian]{babel}
\usepackage{amsfonts} 
\usepackage{amstext}
\usepackage{amssymb}
\usepackage{amsthm}
\usepackage{graphicx} 
\usepackage{subfig}
\usepackage{color}
\usepackage{svg}
\usepackage{epstopdf}
\usepackage[unicode]{hyperref}
\usepackage[nottoc]{tocbibind} 
\usepackage{verbatim}
\usepackage{listings}
\usepackage{indentfirst}
\usepackage{commath}
\usepackage[colorinlistoftodos, prependcaption]{todonotes}
\usepackage{multirow}
\usepackage{algorithmic}
\usepackage{algorithm}
\usepackage{amsmath}
\DeclareMathOperator*{\argmax}{arg\,max}

\author{Гончаров Илья Викторович \\ 421 гр.}
\title{Задача оптимизации построения расписания выполнения задач на множестве процессов}
\date{\today}

\begin{document}
    \maketitle
	\section*{Формальная постановка}
	\subsection*{Дано:}
	
	\begin{itemize}
	    \item \( N \) — количество независимых работ;
	    \item \( M \) — количество процессоров;
	    \item \( t_i \), \( i = 1, 2, \dots, N \) — время выполнения каждой работы \( i \), \( t_i > 0 \).
	\end{itemize}
	
	\subsection*{Требуется:}
	
	Построить расписание выполнения всех
	\( N \) работ на \( M \) процессорах без прерываний, минимизируя указанный критерий.

	\subsection*{Расписание}

	Расписание представлено в виде булевой матрицы $X$ размера $ M \times N $, в которой каждому процессору соответствует строка матрицы, а каждому столбцу - задача, которую необходимо выполнить. Значение 1 стоит в клетке с координатами (i, j), если i-ому процессору назначена на выполнение j-ая задача, иначе в этой клетке стоит 0.

   Время начала выполнения работы определяется формулой $$t^{start}_i = \sum^{b}_{k=1}t^{finish}_{i-1},$$ где b - количество работ на том же процессоре, на котором выполняется работа i, которые выполняются до работы i. Время окончания работы определяется формулой $$t_i^{finish} = t^{start}_i + t_i$$ 

    \begin{flushleft}
	   Приведем пример расписания. 
    \end{flushleft}
	\begin{table}[h]
	\begin{tabular}{|p{1.8cm}||p{1.8cm}|p{1.8cm}|p{1.8cm}|p{1.8cm}|p{1.8cm}|} 
		\hline
		& 1 & 2 & 3 & \ldots & $ N $ \\ 
		\hline
        \hline
			1 & 0 & 1 & 0 & \ldots & 0 \\ 
		\hline
			2 & 1 & 0 & 0 & \ldots & 0 \\ 
		\hline
			3 & 0 & 0 & 1 & \ldots & 0 \\ 
		\hline
			4 & 0 & 0 & 0 & \ldots & 0 \\ 
		\hline
			\ldots & \ldots  & \ldots  & \ldots  & \ldots & \ldots  \\ 	
		\hline
			$M-1$ & 0 & 0 & 0 & \ldots & 1 \\ 
		\hline
			$M$ & 0 & 0 & 0 & \ldots & 0\\ 
		\hline
	\end{tabular}
		\caption{Пример расписания}
		\label{rasp}
	\end{table}

    \newpage

 	В данном примере:
	\begin{itemize}
		\item 1-ая задача выполняется 2 процессором; 
		\item 2-ая задача выполняется 1 процессором;
		\item 3-ая задача выполняется 3 процессором;
		\item N-ая задача выполняется M - 1 процессором.
	\end{itemize}

	\subsection*{Минимизируемый критерий:}
		
	Критерий выбирается на основе значения контрольной суммы CRC32 от фамилии и инициалов:
	\begin{itemize}
		\label{cr1}
	    \item Остаток от деления CRC32 на 2 равен 1: минимизируется \( K_1 \);
	    \item Остаток от деления CRC32 на 2 равен 0: минимизируется \( K_2 \).
	\end{itemize}
	Мой критерий --- К2
	
    \begin{flushleft}
	Критерий \( K_2 \):       
    \end{flushleft}
	\[
	K_2 = \sum^N_{i=1}t^{finish}_i,
	\]
	
	\subsection*{Ограничения на корректность расписания:}
	
	Расписание должно удовлетворять следующим условиям:
	\begin{itemize}
	    \item $\sum_{i = 1}^{M} x_{ij} = 1, \forall j \in [1, N]$ --- т.е каждая работа назначена только на один процессор и все работы распределены по процессорам.
	    \item $t_i^{finish} - t^{start}_i = t_i$ --- прерываний при выполнении работы нет, $t_{i} = const$ --- время выполнения каждой работы фиксировано
        \item $t_j^{finish} = t^{start}_{k}$ где $t_j$ и $t_k$ соответствуют $x_{ij}$ и $x_{ik}$, причем $j < k$.
	\end{itemize}
\end{document}